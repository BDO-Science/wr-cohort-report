% Options for packages loaded elsewhere
\PassOptionsToPackage{unicode}{hyperref}
\PassOptionsToPackage{hyphens}{url}
%
\documentclass[
]{book}
\usepackage{amsmath,amssymb}
\usepackage{iftex}
\ifPDFTeX
  \usepackage[T1]{fontenc}
  \usepackage[utf8]{inputenc}
  \usepackage{textcomp} % provide euro and other symbols
\else % if luatex or xetex
  \usepackage{unicode-math} % this also loads fontspec
  \defaultfontfeatures{Scale=MatchLowercase}
  \defaultfontfeatures[\rmfamily]{Ligatures=TeX,Scale=1}
\fi
\usepackage{lmodern}
\ifPDFTeX\else
  % xetex/luatex font selection
\fi
% Use upquote if available, for straight quotes in verbatim environments
\IfFileExists{upquote.sty}{\usepackage{upquote}}{}
\IfFileExists{microtype.sty}{% use microtype if available
  \usepackage[]{microtype}
  \UseMicrotypeSet[protrusion]{basicmath} % disable protrusion for tt fonts
}{}
\makeatletter
\@ifundefined{KOMAClassName}{% if non-KOMA class
  \IfFileExists{parskip.sty}{%
    \usepackage{parskip}
  }{% else
    \setlength{\parindent}{0pt}
    \setlength{\parskip}{6pt plus 2pt minus 1pt}}
}{% if KOMA class
  \KOMAoptions{parskip=half}}
\makeatother
\usepackage{xcolor}
\usepackage{color}
\usepackage{fancyvrb}
\newcommand{\VerbBar}{|}
\newcommand{\VERB}{\Verb[commandchars=\\\{\}]}
\DefineVerbatimEnvironment{Highlighting}{Verbatim}{commandchars=\\\{\}}
% Add ',fontsize=\small' for more characters per line
\usepackage{framed}
\definecolor{shadecolor}{RGB}{248,248,248}
\newenvironment{Shaded}{\begin{snugshade}}{\end{snugshade}}
\newcommand{\AlertTok}[1]{\textcolor[rgb]{0.94,0.16,0.16}{#1}}
\newcommand{\AnnotationTok}[1]{\textcolor[rgb]{0.56,0.35,0.01}{\textbf{\textit{#1}}}}
\newcommand{\AttributeTok}[1]{\textcolor[rgb]{0.13,0.29,0.53}{#1}}
\newcommand{\BaseNTok}[1]{\textcolor[rgb]{0.00,0.00,0.81}{#1}}
\newcommand{\BuiltInTok}[1]{#1}
\newcommand{\CharTok}[1]{\textcolor[rgb]{0.31,0.60,0.02}{#1}}
\newcommand{\CommentTok}[1]{\textcolor[rgb]{0.56,0.35,0.01}{\textit{#1}}}
\newcommand{\CommentVarTok}[1]{\textcolor[rgb]{0.56,0.35,0.01}{\textbf{\textit{#1}}}}
\newcommand{\ConstantTok}[1]{\textcolor[rgb]{0.56,0.35,0.01}{#1}}
\newcommand{\ControlFlowTok}[1]{\textcolor[rgb]{0.13,0.29,0.53}{\textbf{#1}}}
\newcommand{\DataTypeTok}[1]{\textcolor[rgb]{0.13,0.29,0.53}{#1}}
\newcommand{\DecValTok}[1]{\textcolor[rgb]{0.00,0.00,0.81}{#1}}
\newcommand{\DocumentationTok}[1]{\textcolor[rgb]{0.56,0.35,0.01}{\textbf{\textit{#1}}}}
\newcommand{\ErrorTok}[1]{\textcolor[rgb]{0.64,0.00,0.00}{\textbf{#1}}}
\newcommand{\ExtensionTok}[1]{#1}
\newcommand{\FloatTok}[1]{\textcolor[rgb]{0.00,0.00,0.81}{#1}}
\newcommand{\FunctionTok}[1]{\textcolor[rgb]{0.13,0.29,0.53}{\textbf{#1}}}
\newcommand{\ImportTok}[1]{#1}
\newcommand{\InformationTok}[1]{\textcolor[rgb]{0.56,0.35,0.01}{\textbf{\textit{#1}}}}
\newcommand{\KeywordTok}[1]{\textcolor[rgb]{0.13,0.29,0.53}{\textbf{#1}}}
\newcommand{\NormalTok}[1]{#1}
\newcommand{\OperatorTok}[1]{\textcolor[rgb]{0.81,0.36,0.00}{\textbf{#1}}}
\newcommand{\OtherTok}[1]{\textcolor[rgb]{0.56,0.35,0.01}{#1}}
\newcommand{\PreprocessorTok}[1]{\textcolor[rgb]{0.56,0.35,0.01}{\textit{#1}}}
\newcommand{\RegionMarkerTok}[1]{#1}
\newcommand{\SpecialCharTok}[1]{\textcolor[rgb]{0.81,0.36,0.00}{\textbf{#1}}}
\newcommand{\SpecialStringTok}[1]{\textcolor[rgb]{0.31,0.60,0.02}{#1}}
\newcommand{\StringTok}[1]{\textcolor[rgb]{0.31,0.60,0.02}{#1}}
\newcommand{\VariableTok}[1]{\textcolor[rgb]{0.00,0.00,0.00}{#1}}
\newcommand{\VerbatimStringTok}[1]{\textcolor[rgb]{0.31,0.60,0.02}{#1}}
\newcommand{\WarningTok}[1]{\textcolor[rgb]{0.56,0.35,0.01}{\textbf{\textit{#1}}}}
\usepackage{longtable,booktabs,array}
\usepackage{calc} % for calculating minipage widths
% Correct order of tables after \paragraph or \subparagraph
\usepackage{etoolbox}
\makeatletter
\patchcmd\longtable{\par}{\if@noskipsec\mbox{}\fi\par}{}{}
\makeatother
% Allow footnotes in longtable head/foot
\IfFileExists{footnotehyper.sty}{\usepackage{footnotehyper}}{\usepackage{footnote}}
\makesavenoteenv{longtable}
\usepackage{graphicx}
\makeatletter
\def\maxwidth{\ifdim\Gin@nat@width>\linewidth\linewidth\else\Gin@nat@width\fi}
\def\maxheight{\ifdim\Gin@nat@height>\textheight\textheight\else\Gin@nat@height\fi}
\makeatother
% Scale images if necessary, so that they will not overflow the page
% margins by default, and it is still possible to overwrite the defaults
% using explicit options in \includegraphics[width, height, ...]{}
\setkeys{Gin}{width=\maxwidth,height=\maxheight,keepaspectratio}
% Set default figure placement to htbp
\makeatletter
\def\fps@figure{htbp}
\makeatother
\setlength{\emergencystretch}{3em} % prevent overfull lines
\providecommand{\tightlist}{%
  \setlength{\itemsep}{0pt}\setlength{\parskip}{0pt}}
\setcounter{secnumdepth}{5}
\usepackage{booktabs}
\ifLuaTeX
  \usepackage{selnolig}  % disable illegal ligatures
\fi
\usepackage[]{natbib}
\bibliographystyle{plainnat}
\IfFileExists{bookmark.sty}{\usepackage{bookmark}}{\usepackage{hyperref}}
\IfFileExists{xurl.sty}{\usepackage{xurl}}{} % add URL line breaks if available
\urlstyle{same}
\hypersetup{
  pdftitle={Brood Year 2021 Winter-Run Chinook Salmon Report},
  pdfauthor={Catarina Pien (Bureau of Reclamation)},
  hidelinks,
  pdfcreator={LaTeX via pandoc}}

\title{Brood Year 2021 Winter-Run Chinook Salmon Report}
\author{Catarina Pien (Bureau of Reclamation)}
\date{2023-08-30}

\usepackage{amsthm}
\newtheorem{theorem}{Theorem}[chapter]
\newtheorem{lemma}{Lemma}[chapter]
\newtheorem{corollary}{Corollary}[chapter]
\newtheorem{proposition}{Proposition}[chapter]
\newtheorem{conjecture}{Conjecture}[chapter]
\theoremstyle{definition}
\newtheorem{definition}{Definition}[chapter]
\theoremstyle{definition}
\newtheorem{example}{Example}[chapter]
\theoremstyle{definition}
\newtheorem{exercise}{Exercise}[chapter]
\theoremstyle{definition}
\newtheorem{hypothesis}{Hypothesis}[chapter]
\theoremstyle{remark}
\newtheorem*{remark}{Remark}
\newtheorem*{solution}{Solution}
\begin{document}
\maketitle

{
\setcounter{tocdepth}{1}
\tableofcontents
}
\hypertarget{about}{%
\chapter{About}\label{about}}

We summarize environmental and habitat conditions in 2021 and assess the 2021 brood year of Sacramento winter-run Chinook salmon (WRCS; \emph{Oncorhynchus tshawytscha}) (\textbf{BY 2021}). We used data available online to generate this report. This report follows the format of the BY 2019 WRCS Report written by Anchor QEA (@ref(\url{https://www.anchorqea.com/news/brood-year-2019-winter-run-chinook-salmon-operations-and-monitoring-assessment/})). The assessment was in collaboration with the \texttt{Sacramento\ River\ Science\ Partnership}.

\hypertarget{wr-chinook-salmon-life-history}{%
\section{WR Chinook Salmon Life History}\label{wr-chinook-salmon-life-history}}

Sacramento River WRCS begin their spawning migration in November, traveling from the San Francisco Bay to the upper Sacramento River, and spawning between mid-April to August. Juvenile WRCS emigrate downstream between July-March, and are present in the Delta between September-June.

\hypertarget{wr-chinook-salmon-threats}{%
\section{WR Chinook Salmon Threats}\label{wr-chinook-salmon-threats}}

WRCS historically spawned in cold-water reaches of the McCloud, Pit, and Sacramento Rivers. The construction of Shasta and Keswick Dams blocked WRCS from returning to the cooler spawning grounds, and the population is now limited to spawning below Keswick Dam, which experiences higher water temperatures and lower flows.

WRCS were listed under the California Endangered Species Act (CESA) in 1989, and were listed under the Federal Endangered Species Act as endangered on January 4, 1994.

\hypertarget{spatial-distribution}{%
\section{Spatial Distribution}\label{spatial-distribution}}

\begin{figure}
\centering
\includegraphics{figures/map_sac_river_delta_bay.png}
\caption{``distribution map''}
\end{figure}

\hypertarget{conceptual-model}{%
\section{Conceptual Model}\label{conceptual-model}}

Metrics selected in this report are based on a conceptual model developed by Windell et al.~(2017).

\begin{figure}
\centering
\includegraphics{figures/WRCM.png}
\caption{``conceptual model''}
\end{figure}

\hypertarget{references}{%
\section{References}\label{references}}

\begin{itemize}
\tightlist
\item
  \url{https://wildlife.ca.gov/Conservation/Fishes/Chinook-Salmon/Winter-run}
\item
  Moyle P.B. 2002. Inland Fishes of California, University of California Press.
\item
  National Marine Fisheries Service (NMFS). 2014. Recovery Plan for Evolutionarily Significant Units of Sacramento River Winter-run Chinook Salmon and Central Valley Spring-run Chinook Salmon and the Distinct population Segment of California Central Valley Steelhead. California Central Valley Area Office, July 2014.
\item
  Windell, S., P.L. Brandes, J.L. Conrad, J.W. Ferguson, P.A.L. Goertler, B.N. Harvey, J. Heublein, J.A. Israel, D.W. Kratville, J.E. Kirsch, R.W. Perry, J. Pisciotto, W.R. Poytress, K. Reece, B.G. Swart, and R.C. Johnson, 2017. Scientific Framework for Assessing Factors Influencing Endangered Sacramento River Winter­Run Chinook Salmon (Oncorhynchus tshawytscha) Across the Life Cycle. NOAA Technical Memorandum NMFS. NOAA-TM-NMFS-SWFSC-586. August 2017. Available at: \url{https://watershed.ucdavis.edu/files/biblio/NOAA-TM-NMFS-SWFSC-586_Final.pdf}
\end{itemize}

\hypertarget{adults}{%
\chapter{Adults}\label{adults}}

This section describes environmental attributes associated with and responses during the adult life stage (ocean harvest, migration, spawning)

\hypertarget{habitat-attributes}{%
\section{Habitat Attributes}\label{habitat-attributes}}

\begin{enumerate}
\def\labelenumi{\arabic{enumi}.}
\item
  Hatchery Influence (Proportion of hatchery return)
\item
  Hatchery Pathogens/Disease
\item
  In-River Pathogens/Disease
\item
  Spawning Habitat Capacity (SIT model)
\end{enumerate}

\hypertarget{wq}{%
\section{Environmental Drivers}\label{wq}}

\begin{enumerate}
\def\labelenumi{\arabic{enumi}.}
\tightlist
\item
  Water Temperature
\end{enumerate}

Water temperatures were warmer than average in 2020 (Figure \ref{fig:historicalwtemp-fig}).

\begin{figure}
\centering
\includegraphics{_main_files/figure-latex/BSFwtemp-fig-1.pdf}
\caption{\label{fig:BSFwtemp-fig}Sacramento River Water Temperature at Ball's Ferry Bridge TCP (BSF).}
\end{figure}

\begin{figure}
\centering
\includegraphics{_main_files/figure-latex/historicalwtemp-fig-1.pdf}
\caption{\label{fig:historicalwtemp-fig}Historical Comparison of Sacramento River Water Temperature at Clear Creek (CCR) and Balls Ferry Bridge (BSF).}
\end{figure}

\hypertarget{biological-response}{%
\section{Biological Response}\label{biological-response}}

\begin{enumerate}
\def\labelenumi{\arabic{enumi}.}
\item
  Ocean Harvest Rates (PFMC)
\item
  Adults to Hatchery (GrandTab)
\item
  Estimated Total Mainstem In-River Spawners of Natural and Hatchery Origin (GrandTab)
\end{enumerate}

\begin{itemize}
\tightlist
\item
  Downstream RBDD
\item
  Upstream RBDD
\item
  Clear Creek
\item
  Battle Creek
\end{itemize}

\begin{enumerate}
\def\labelenumi{\arabic{enumi}.}
\setcounter{enumi}{3}
\tightlist
\item
  Adult Condition (Carcass Surveys)
\end{enumerate}

\begin{itemize}
\tightlist
\item
  Male Fork Lengths (Histogram)
\item
  Female Fork Lengths (Histogram)
\item
  Age Distribution
\item
  Thiamine Deficiency
\item
  Pre-spawn mortality (in-text current year and 10-yr average, compare with other years)
\end{itemize}

\begin{enumerate}
\def\labelenumi{\arabic{enumi}.}
\setcounter{enumi}{4}
\tightlist
\item
  Spawn Timing (Carcass Surveys)
\end{enumerate}

\begin{itemize}
\tightlist
\item
  Percent spawning by week - line plot of percent of carcasses by week (current year, 10 year and 20 year average)
\item
  Peak spawning week - line plot of peak spawning week by year (annual, rolling 5-year avg, 10-year avg)
\end{itemize}

\begin{enumerate}
\def\labelenumi{\arabic{enumi}.}
\setcounter{enumi}{5}
\tightlist
\item
  Number of Winter-Run Chinook Salmon Redds (aerial redd surveys, Calfish)
\end{enumerate}

\begin{itemize}
\tightlist
\item
  line plot of count by year with average horizontal line
\end{itemize}

\begin{enumerate}
\def\labelenumi{\arabic{enumi}.}
\setcounter{enumi}{6}
\tightlist
\item
  Distribution of Winter-Run Chinook Salmon Redds - location of redds (aerial redd surveys, Calfish)
\end{enumerate}

\begin{itemize}
\tightlist
\item
  map of locations
\item
  bar plot of percent redds by location (historical and this year)
\item
  bar plot of female spawner carcasses by location
\end{itemize}

\begin{enumerate}
\def\labelenumi{\arabic{enumi}.}
\setcounter{enumi}{7}
\tightlist
\item
  Hatchery Fecundity (JPE Letters)
\end{enumerate}

\begin{itemize}
\tightlist
\item
  in text, comparison with previous 10 years
\end{itemize}

\begin{enumerate}
\def\labelenumi{\arabic{enumi}.}
\setcounter{enumi}{8}
\tightlist
\item
  Hatchery Influence (Hatchery report?)
\end{enumerate}

\begin{itemize}
\tightlist
\item
  in text percentage of hatchery fish/natural spawning pop
\end{itemize}

\begin{enumerate}
\def\labelenumi{\arabic{enumi}.}
\setcounter{enumi}{9}
\tightlist
\item
  Cohort Replacement Rate (GrandTab)
\end{enumerate}

\begin{itemize}
\tightlist
\item
  line plot of CRR by year
\end{itemize}

\hypertarget{egg-to-fry-emergence}{%
\chapter{Egg to Fry Emergence}\label{egg-to-fry-emergence}}

This section describes environmental attributes associated with and responses during the egg-to-fry life stage.

\hypertarget{habitat-attributes-1}{%
\section{Habitat Attributes}\label{habitat-attributes-1}}

\begin{enumerate}
\def\labelenumi{\arabic{enumi}.}
\tightlist
\item
  Redd Dewatering
\end{enumerate}

\begin{itemize}
\tightlist
\item
  count by year
\item
  location
\item
  water depth
\item
  flow
\end{itemize}

\hypertarget{environmental-drivers}{%
\section{Environmental Drivers}\label{environmental-drivers}}

\begin{enumerate}
\def\labelenumi{\arabic{enumi}.}
\item
  Air Temperature
\item
  DO
\item
  Shasta Storage/Hydrology
\item
  Water Temperature
\end{enumerate}

\hypertarget{biological-response-1}{%
\section{Biological Response}\label{biological-response-1}}

\begin{enumerate}
\def\labelenumi{\arabic{enumi}.}
\tightlist
\item
  Egg Count
\end{enumerate}

\begin{itemize}
\tightlist
\item
  line plot of potential eggs by year with averages (JPE letter?)
\end{itemize}

\begin{enumerate}
\def\labelenumi{\arabic{enumi}.}
\setcounter{enumi}{1}
\tightlist
\item
  Egg to fry survival
\end{enumerate}

\begin{itemize}
\tightlist
\item
  bar plot of percent survival by year (JPI calculation)

  \begin{itemize}
  \tightlist
  \item
    JPI = fry abundance/total viable eggs in JPE letter
  \end{itemize}
\item
  egg to fry survival by number of mainstem in-river spawners (JPI calculation)
\item
  egg-to-fry survival from fish model (fish model)
\end{itemize}

\begin{enumerate}
\def\labelenumi{\arabic{enumi}.}
\setcounter{enumi}{2}
\tightlist
\item
  Emergence Timing (fish model)
\end{enumerate}

\hypertarget{upper-sacramento-juveniles}{%
\chapter{Upper Sacramento Juveniles}\label{upper-sacramento-juveniles}}

This section describes environmental attributes associated with and responses during the out-migrating juvenile life stage in the Upper Sacramento River.

\hypertarget{habitat-attributes-2}{%
\section{Habitat Attributes}\label{habitat-attributes-2}}

\begin{enumerate}
\def\labelenumi{\arabic{enumi}.}
\item
  Hatchery Influence (Juvenile Releases)
\item
  Juvenile Stranding
\item
  Pathogens/Disease
\item
  Contaminants
\item
  Habitat Capacity (Floodplain Connectivity)
\item
  Habitat Capacity: Depth/Shallow Water
\item
  In-Stream Habitat Capacity
\end{enumerate}

\hypertarget{environmental-drivers-1}{%
\section{Environmental Drivers}\label{environmental-drivers-1}}

\begin{enumerate}
\def\labelenumi{\arabic{enumi}.}
\item
  Shasta Storage/Hydrology
\item
  Flows: Migration Cues
\item
  Flows at Keswick
\end{enumerate}

Flows at Keswick were lower in 2022 (Figure \ref{fig:KWKflow-fig}).
\includegraphics{_main_files/figure-latex/KWKflow-fig-1.pdf}

\begin{enumerate}
\def\labelenumi{\arabic{enumi}.}
\setcounter{enumi}{3}
\tightlist
\item
  Turbidity and DO
  \includegraphics{_main_files/figure-latex/DOTurb-fig-1.pdf}
\item
  Water Temperature
\end{enumerate}

\hypertarget{biological-response-2}{%
\section{Biological Response}\label{biological-response-2}}

\begin{enumerate}
\def\labelenumi{\arabic{enumi}.}
\tightlist
\item
  Fry abundance (Fry-equivalent JPI)
\end{enumerate}

\begin{itemize}
\tightlist
\item
  By year
\item
  RBDD RST Data
\end{itemize}

\begin{enumerate}
\def\labelenumi{\arabic{enumi}.}
\setcounter{enumi}{1}
\tightlist
\item
  Condition/ Growth
\end{enumerate}

\begin{itemize}
\tightlist
\item
  Fork length by year
\end{itemize}

\begin{enumerate}
\def\labelenumi{\arabic{enumi}.}
\setcounter{enumi}{2}
\tightlist
\item
  Migration Timing
\end{enumerate}

\begin{itemize}
\tightlist
\item
  RBDD RST Data
\end{itemize}

\begin{enumerate}
\def\labelenumi{\arabic{enumi}.}
\setcounter{enumi}{3}
\tightlist
\item
  Fry-to-Smolt Survival
\end{enumerate}

\begin{itemize}
\tightlist
\item
  Model?
\end{itemize}

\hypertarget{middle-and-lower-sacramento-juveniles}{%
\chapter{Middle and Lower Sacramento Juveniles}\label{middle-and-lower-sacramento-juveniles}}

This section describes environmental attributes associated with and responses during the out-migrating juvenile life stage in the Lower and Middle Sacramento River.

\hypertarget{habitat-attributes-3}{%
\section{Habitat Attributes}\label{habitat-attributes-3}}

\begin{enumerate}
\def\labelenumi{\arabic{enumi}.}
\item
  Habitat Capacity (Floodplain Connectivity)
\item
  Habitat Capacity: Depth/Shallow Water
\item
  In-Stream Habitat Capacity
\end{enumerate}

\hypertarget{environmental-drivers-2}{%
\section{Environmental Drivers}\label{environmental-drivers-2}}

\begin{enumerate}
\def\labelenumi{\arabic{enumi}.}
\item
  Shasta Storage/Hydrology
\item
  Flows: Migration Cues
\item
  Flows: In-River Flows
\item
  Turbidity
\item
  Water Temperature
\end{enumerate}

\hypertarget{biological-response-3}{%
\section{Biological Response}\label{biological-response-3}}

Monitoring Sources for abundance, growth/size, migration timing/duration

\begin{itemize}
\tightlist
\item
  Sac Trawl
\item
  Tisdale Weir
\item
  Knights Landing
\item
  GCID
\item
  DJFMP
\item
  Yolo Bypass
\item
  Chipps Island Trawl (Exit)
\item
  Genetic (Chipps, SWP/CVP, Knights Landing, Yolo Bypass)
\end{itemize}

\begin{enumerate}
\def\labelenumi{\arabic{enumi}.}
\tightlist
\item
  Abundance (Count) (IEP Monitoring)
\end{enumerate}

\begin{itemize}
\tightlist
\item
  Natural JPE
\item
  Hatchery JPE
\item
  SacPAS Fish Model (emerged fry)
\end{itemize}

\begin{enumerate}
\def\labelenumi{\arabic{enumi}.}
\setcounter{enumi}{1}
\tightlist
\item
  Condition
\end{enumerate}

\begin{itemize}
\tightlist
\item
  Growth/ Size
\end{itemize}

\begin{enumerate}
\def\labelenumi{\arabic{enumi}.}
\setcounter{enumi}{2}
\tightlist
\item
  Migration Timing
\end{enumerate}

\begin{itemize}
\tightlist
\item
  SacPAS style plots of historical and current year?
\end{itemize}

\begin{enumerate}
\def\labelenumi{\arabic{enumi}.}
\setcounter{enumi}{3}
\tightlist
\item
  Survival
\end{enumerate}

\begin{itemize}
\tightlist
\item
  Hatchery real-time: Calfish Track/ERDDAP
\item
  Natural Origin Smolt survival (O Farell et al.~2018)
\item
  Hatchery Origin Smolt survival
\item
  Modeled:
  ** Juvenile: STARS
  ** Fish Model
\end{itemize}

\hypertarget{sacramento-san-joaquin-delta-juveniles}{%
\chapter{Sacramento-San Joaquin Delta Juveniles}\label{sacramento-san-joaquin-delta-juveniles}}

This section describes environmental attributes associated with and responses during the out-migrating juvenile life stage in the Sacramento-San Joaquin Delta.

\hypertarget{habitat-attributes-4}{%
\section{Habitat Attributes}\label{habitat-attributes-4}}

\begin{enumerate}
\def\labelenumi{\arabic{enumi}.}
\tightlist
\item
  Rearing Habitat Capacity (Floodplain Connectivity)
\end{enumerate}

\begin{itemize}
\tightlist
\item
  Weir overtopping
\end{itemize}

\begin{enumerate}
\def\labelenumi{\arabic{enumi}.}
\setcounter{enumi}{1}
\item
  Entrainment Risk
\item
  Food Availability
\end{enumerate}

\hypertarget{environmental-drivers-3}{%
\section{Environmental Drivers}\label{environmental-drivers-3}}

\begin{enumerate}
\def\labelenumi{\arabic{enumi}.}
\item
  Shasta Storage/Hydrology
\item
  Flows: Sacramento River, Delta Outflow
\item
  Flows: Migration Cues and Routing
\item
  Turbidity
\item
  Water Temperature
\item
  DO
\end{enumerate}

\hypertarget{biological-response-4}{%
\section{Biological Response}\label{biological-response-4}}

Monitoring Sources for abundance, growth/size, migration timing/duration

\begin{itemize}
\tightlist
\item
  Sac Trawl
\item
  Tisdale Weir
\item
  Knights Landing
\item
  GCID
\item
  DJFMP
\item
  Yolo Bypass
\item
  Chipps Island Trawl (Exit)
\item
  Genetic (Chipps, SWP/CVP, Knights Landing, Yolo Bypass)
\end{itemize}

\begin{enumerate}
\def\labelenumi{\arabic{enumi}.}
\item
  Abundance (Count) (IEP Monitoring)
\item
  Condition (IEP Monitoring)
\end{enumerate}

\begin{itemize}
\tightlist
\item
  FL
\end{itemize}

\begin{enumerate}
\def\labelenumi{\arabic{enumi}.}
\setcounter{enumi}{2}
\tightlist
\item
  Migration Timing (IEP Monitoring)
\end{enumerate}

\begin{itemize}
\tightlist
\item
  SacPAS style plots of historical and current year?
\end{itemize}

\emph{Chipps Trawl Timing}

\emph{Sac Trawl Timing}

\emph{Sac Beach Seine Timing}

\begin{enumerate}
\def\labelenumi{\arabic{enumi}.}
\setcounter{enumi}{3}
\tightlist
\item
  Migration Duration
\end{enumerate}

\begin{itemize}
\tightlist
\item
  Calfish Track/ERDDAP
\end{itemize}

\begin{enumerate}
\def\labelenumi{\arabic{enumi}.}
\setcounter{enumi}{4}
\item
  Migration Routing
\item
  Survival
\end{enumerate}

\begin{itemize}
\tightlist
\item
  Hatchery real-time: Calfish Track/ERDDAP
\item
  Natural Origin Smolt survival (O Farell et al.~2018)
\item
  Hatchery Origin Smolt survival
\item
  Modeled:
  ** Juvenile: STARS
  ** Fish Model
\item
  Survival to Delta: Production (Hatchery JPE, Modeled JPE)
\end{itemize}

\begin{enumerate}
\def\labelenumi{\arabic{enumi}.}
\setcounter{enumi}{6}
\tightlist
\item
  Loss and Salvage (Salvage)
\end{enumerate}

\begin{itemize}
\tightlist
\item
  Take Limit
\item
  Model
\end{itemize}

\hypertarget{abbreviations}{%
\chapter{Abbreviations}\label{abbreviations}}

CM = Conceptual Model
WRCS = Winter Run Chinook Salmon

\hypertarget{useful-info}{%
\chapter{Useful info}\label{useful-info}}

\hypertarget{parts}{%
\section{Parts}\label{parts}}

You can add parts to organize one or more book chapters together. Parts can be inserted at the top of an .Rmd file, before the first-level chapter heading in that same file.

Add a numbered part: \texttt{\#\ (PART)\ Act\ one\ \{-\}} (followed by \texttt{\#\ A\ chapter})

Add an unnumbered part: \texttt{\#\ (PART\textbackslash{}*)\ Act\ one\ \{-\}} (followed by \texttt{\#\ A\ chapter})

Add an appendix as a special kind of un-numbered part: \texttt{\#\ (APPENDIX)\ Other\ stuff\ \{-\}} (followed by \texttt{\#\ A\ chapter}). Chapters in an appendix are prepended with letters instead of numbers.

\hypertarget{footnotes-and-citations}{%
\section{Footnotes and citations}\label{footnotes-and-citations}}

\hypertarget{footnotes}{%
\subsection{Footnotes}\label{footnotes}}

Footnotes are put inside the square brackets after a caret \texttt{\^{}{[}{]}}. Like this one \footnote{This is a footnote.}.

\hypertarget{citations}{%
\subsection{Citations}\label{citations}}

Reference items in your bibliography file(s) using \texttt{@key}.

For example, we are using the \textbf{bookdown} package \citep{R-bookdown} (check out the last code chunk in index.Rmd to see how this citation key was added) in this sample book, which was built on top of R Markdown and \textbf{knitr} \citep{xie2015} (this citation was added manually in an external file book.bib).
Note that the \texttt{.bib} files need to be listed in the index.Rmd with the YAML \texttt{bibliography} key.

The RStudio Visual Markdown Editor can also make it easier to insert citations: \url{https://rstudio.github.io/visual-markdown-editing/\#/citations}

\hypertarget{blocks}{%
\section{Blocks}\label{blocks}}

\hypertarget{equations}{%
\subsection{Equations}\label{equations}}

Here is an equation.

\begin{equation} 
  f\left(k\right) = \binom{n}{k} p^k\left(1-p\right)^{n-k}
  \label{eq:binom}
\end{equation}

You may refer to using \texttt{\textbackslash{}@ref(eq:binom)}, like see Equation \eqref{eq:binom}.

\hypertarget{theorems-and-proofs}{%
\subsection{Theorems and proofs}\label{theorems-and-proofs}}

Labeled theorems can be referenced in text using \texttt{\textbackslash{}@ref(thm:tri)}, for example, check out this smart theorem \ref{thm:tri}.

\begin{theorem}
\protect\hypertarget{thm:tri}{}\label{thm:tri}For a right triangle, if \(c\) denotes the \emph{length} of the hypotenuse
and \(a\) and \(b\) denote the lengths of the \textbf{other} two sides, we have
\[a^2 + b^2 = c^2\]
\end{theorem}

Read more here \url{https://bookdown.org/yihui/bookdown/markdown-extensions-by-bookdown.html}.

\hypertarget{callout-blocks}{%
\subsection{Callout blocks}\label{callout-blocks}}

The R Markdown Cookbook provides more help on how to use custom blocks to design your own callouts: \url{https://bookdown.org/yihui/rmarkdown-cookbook/custom-blocks.html}

\hypertarget{cross}{%
\section{Cross-references}\label{cross}}

Cross-references make it easier for your readers to find and link to elements in your book.

\hypertarget{chapters-and-sub-chapters}{%
\subsection{Chapters and sub-chapters}\label{chapters-and-sub-chapters}}

There are two steps to cross-reference any heading:

\begin{enumerate}
\def\labelenumi{\arabic{enumi}.}
\tightlist
\item
  Label the heading: \texttt{\#\ Hello\ world\ \{\#nice-label\}}.

  \begin{itemize}
  \tightlist
  \item
    Leave the label off if you like the automated heading generated based on your heading title: for example, \texttt{\#\ Hello\ world} = \texttt{\#\ Hello\ world\ \{\#hello-world\}}.
  \item
    To label an un-numbered heading, use: \texttt{\#\ Hello\ world\ \{-\#nice-label\}} or \texttt{\{\#\ Hello\ world\ .unnumbered\}}.
  \end{itemize}
\item
  Next, reference the labeled heading anywhere in the text using \texttt{\textbackslash{}@ref(nice-label)}; for example, please see Chapter \ref{cross}.

  \begin{itemize}
  \tightlist
  \item
    If you prefer text as the link instead of a numbered reference use: \protect\hyperlink{cross}{any text you want can go here}.
  \end{itemize}
\end{enumerate}

\hypertarget{captioned-figures-and-tables}{%
\subsection{Captioned figures and tables}\label{captioned-figures-and-tables}}

Figures and tables \emph{with captions} can also be cross-referenced from elsewhere in your book using \texttt{\textbackslash{}@ref(fig:chunk-label)} and \texttt{\textbackslash{}@ref(tab:chunk-label)}, respectively.

See Figure \ref{fig:nice-fig}.

\begin{Shaded}
\begin{Highlighting}[]
\FunctionTok{par}\NormalTok{(}\AttributeTok{mar =} \FunctionTok{c}\NormalTok{(}\DecValTok{4}\NormalTok{, }\DecValTok{4}\NormalTok{, .}\DecValTok{1}\NormalTok{, .}\DecValTok{1}\NormalTok{))}
\FunctionTok{plot}\NormalTok{(pressure, }\AttributeTok{type =} \StringTok{\textquotesingle{}b\textquotesingle{}}\NormalTok{, }\AttributeTok{pch =} \DecValTok{19}\NormalTok{)}
\end{Highlighting}
\end{Shaded}

\begin{figure}

{\centering \includegraphics[width=0.8\linewidth]{_main_files/figure-latex/nice-fig-1} 

}

\caption{Here is a nice figure!}\label{fig:nice-fig}
\end{figure}

Don't miss Table \ref{tab:nice-tab}.

\begin{Shaded}
\begin{Highlighting}[]
\NormalTok{knitr}\SpecialCharTok{::}\FunctionTok{kable}\NormalTok{(}
  \FunctionTok{head}\NormalTok{(pressure, }\DecValTok{10}\NormalTok{), }\AttributeTok{caption =} \StringTok{\textquotesingle{}Here is a nice table!\textquotesingle{}}\NormalTok{,}
  \AttributeTok{booktabs =} \ConstantTok{TRUE}
\NormalTok{)}
\end{Highlighting}
\end{Shaded}

\begin{table}

\caption{\label{tab:nice-tab}Here is a nice table!}
\centering
\begin{tabular}[t]{rr}
\toprule
temperature & pressure\\
\midrule
0 & 0.0002\\
20 & 0.0012\\
40 & 0.0060\\
60 & 0.0300\\
80 & 0.0900\\
\addlinespace
100 & 0.2700\\
120 & 0.7500\\
140 & 1.8500\\
160 & 4.2000\\
180 & 8.8000\\
\bottomrule
\end{tabular}
\end{table}

\hypertarget{sharing-your-book}{%
\section{Sharing your book}\label{sharing-your-book}}

\hypertarget{publishing}{%
\subsection{Publishing}\label{publishing}}

HTML books can be published online, see: \url{https://bookdown.org/yihui/bookdown/publishing.html}

\hypertarget{pages}{%
\subsection{404 pages}\label{pages}}

By default, users will be directed to a 404 page if they try to access a webpage that cannot be found. If you'd like to customize your 404 page instead of using the default, you may add either a \texttt{\_404.Rmd} or \texttt{\_404.md} file to your project root and use code and/or Markdown syntax.

\hypertarget{metadata-for-sharing}{%
\subsection{Metadata for sharing}\label{metadata-for-sharing}}

Bookdown HTML books will provide HTML metadata for social sharing on platforms like Twitter, Facebook, and LinkedIn, using information you provide in the \texttt{index.Rmd} YAML. To setup, set the \texttt{url} for your book and the path to your \texttt{cover-image} file. Your book's \texttt{title} and \texttt{description} are also used.

This \texttt{gitbook} uses the same social sharing data across all chapters in your book- all links shared will look the same.

Specify your book's source repository on GitHub using the \texttt{edit} key under the configuration options in the \texttt{\_output.yml} file, which allows users to suggest an edit by linking to a chapter's source file.

Read more about the features of this output format here:

\url{https://pkgs.rstudio.com/bookdown/reference/gitbook.html}

Or use:

\begin{Shaded}
\begin{Highlighting}[]
\NormalTok{?bookdown}\SpecialCharTok{::}\NormalTok{gitbook}
\end{Highlighting}
\end{Shaded}

\hypertarget{render-book}{%
\section{Render book}\label{render-book}}

You can render the HTML version of this example book without changing anything:

\begin{enumerate}
\def\labelenumi{\arabic{enumi}.}
\item
  Find the \textbf{Build} pane in the RStudio IDE, and
\item
  Click on \textbf{Build Book}, then select your output format, or select ``All formats'' if you'd like to use multiple formats from the same book source files.
\end{enumerate}

Or build the book from the R console:

\begin{Shaded}
\begin{Highlighting}[]
\NormalTok{bookdown}\SpecialCharTok{::}\FunctionTok{render\_book}\NormalTok{()}
\end{Highlighting}
\end{Shaded}

To render this example to PDF as a \texttt{bookdown::pdf\_book}, you'll need to install XeLaTeX. You are recommended to install TinyTeX (which includes XeLaTeX): \url{https://yihui.org/tinytex/}.

\hypertarget{preview-book}{%
\section{Preview book}\label{preview-book}}

As you work, you may start a local server to live preview this HTML book. This preview will update as you edit the book when you save individual .Rmd files. You can start the server in a work session by using the RStudio add-in ``Preview book'', or from the R console:

\begin{Shaded}
\begin{Highlighting}[]
\NormalTok{bookdown}\SpecialCharTok{::}\FunctionTok{serve\_book}\NormalTok{()}
\end{Highlighting}
\end{Shaded}

\hypertarget{footnotes-and-citations-1}{%
\section{Footnotes and citations}\label{footnotes-and-citations-1}}

\hypertarget{footnotes-1}{%
\subsection{Footnotes}\label{footnotes-1}}

Footnotes are put inside the square brackets after a caret \texttt{\^{}{[}{]}}. Like this one \footnote{This is a footnote.}.

\hypertarget{citations-1}{%
\subsection{Citations}\label{citations-1}}

\begin{itemize}
\tightlist
\item
  \url{https://www.anchorqea.com/news/brood-year-2019-winter-run-chinook-salmon-operations-and-monitoring-assessment/}
\end{itemize}

Reference items in your bibliography file(s) using \texttt{@key}.

For example, we are using the \textbf{bookdown} package \citep{R-bookdown} (check out the last code chunk in index.Rmd to see how this citation key was added) in this sample book, which was built on top of R Markdown and \textbf{knitr} \citep{xie2015} (this citation was added manually in an external file book.bib). Note that the \texttt{.bib} files need to be listed in the index.Rmd with the YAML \texttt{bibliography} key.

The RStudio Visual Markdown Editor can also make it easier to insert citations: \url{https://rstudio.github.io/visual-markdown-editing/\#/citations}

  \bibliography{book.bib,packages.bib}

\end{document}
